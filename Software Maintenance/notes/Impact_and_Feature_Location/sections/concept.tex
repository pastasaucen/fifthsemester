\section{Concept Location}

\subsection{Prefactoring}
\begin{flushleft}
Refactor sourcecode without change the bahavior of the program
\end{flushleft}

\begin{flushleft}
If codesmell is discovered, refactoring will be performed.
\end{flushleft}

\begin{flushleft}
	It is also good practice to split of large classes with lots of responsibility (God Classes), into smaller classes to seperate the responsibility.
\end{flushleft}

\subsection{Postfactoring}
\begin{flushleft}
When going over the newly written code, it is important to go though it and eleminate anti-patterns, and lastly clean the code for any mistakes or redundancies.
\end{flushleft}

\subsection{Verification}
\begin{flushleft}
Verifying the newly written code with tests guarantees the correctness of the code.

\subsubsection{Testing}
There are three main tests to be performed:
\begin{itemize}
	\item Functional
	\item Unit
	\item Structual
\end{itemize}

A concept of development is \textbf{Test First development}. This is a technique where a test-specification is written, and the program is coded to match the output/rules of the specification.
\end{flushleft}

\section{Conclusion}
\begin{flushleft}
When programming, commit often. When changes has been made that works, commit. This allows for much clearer version control.

When everything is commited and pushed, it's time to lay out the plan for the next change / sprint.
\end{flushleft}
