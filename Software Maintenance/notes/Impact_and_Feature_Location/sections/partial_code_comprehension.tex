\section{Partial code comprehension}
\begin{flushleft}
Partial code comprehension often appears in larger systems where you have no prior knowledge.
\end{flushleft}

\subsection{Concept Location}
\begin{flushleft}
Concept location is a method where you decide which tool/class you need to edit, and then manually goes through all it's dependencies, mapping out what future changes will affect throughout the system.
\end{flushleft}

\subsection{Concept Triangle}
\begin{flushleft}
	The concept triangle is to take map out concepts in part, ie. \textbf{$<<name>>$}, \textbf{$<<intention>>$} and \textbf{$<<extension>>$}
\end{flushleft}

\subsection{Concept Location Methods}
\begin{itemize}
	\item Human knowledge
		\begin{itemize}
			\item Understandin of the programs domain
		\end{itemize}
	\item Traceability Tools (Featureous, etc)
	\item Dynamic Search (Execution traces)
		\begin{itemize}
			\item Run and go through all features and let the tool map accordingly
		\end{itemize}
	\item GREP Search
		\begin{itemize}
			\item Global Regular Expression Print
		\end{itemize}
\end{itemize}
