\section{Penetration testing}
\begin{flushleft}
Test the security of the running system

Authorized "simulated attack"

It is not a guarantee for security, but it should complement any security strategy, e.g., ongoing Threat Modelling as part of the development/deployment process.

\subsection{Activities}
\begin{itemize}
	\item Getting permission, maybe as part of the job

	\item Gathering Information, e.g.,
		\begin{itemize}
			\item About the company, ip adresses, websites, etc.
			\item Scan ip adresses, hosts services
			\item Fingerprinting sevices
		\end{itemize}
	\item Threat Modelling

	\item Vulnerability Analysis, e.g., connect to collected intel to possible vulnerability

	\item Exploitation, e.g., carefully create proof of concept, demonstrating security problems without harming the target

	\item Reporting, e.g., report and/or fix the problems
\end{itemize}

\subsection{Gather information}
\begin{itemize}
	\item Get IP adresses
\end{itemize}
What secries ports are open
what secvices are running
what versions are the services or OS's running

Passive scanning is checking websites, gathering information to use for social engineering, publically known information.

Active scanning is actively running tools on their machines or systems, nmapping, Nessus(tool) running

Just because a port is open, doesn't mean that it's vulnerable.

Tools for finding vulnerabilities for known systems: exploit-db and Nessus.

\end{flushleft}
