\section{Manufacturing\ processes}
\begin{flushleft}
There are two types of manufacturing, \textbf{continuous} and \textbf{discrete}.

As far as a normal manufacturing project, a mix of continuous and discrete processes is common. When determining whether a process is continuous or discrete, we don't look at where the "raw materials" comes from.

For example, a bakery is usually a continuous process, but whether the manufacturing of the wheat, the sugar, the milk, etc. is one or the other doesn't matter. The materials are a given for the isolated process.

In the following sections, definition and examples will be included.
\end{flushleft}

\subsection{Discrete manufacturing}
\begin{flushleft}
A discrete manufacturing is defined by:
\begin{center}
\textbf{In a discrete process, both the raw materials and the final product are countable.}
\end{center}

A descrete manufacturing could be a car assembly. The whole process is consists of parts, where if an error should occur, the faulty part can be removed and binned, and a new part can be installed, and the process continues.

This process can also be paused over night, and continued the dat after, with no larger effects.
\end{flushleft}

\subsection{Continuous manufacturing}
\begin{flushleft}
A continuous manufacturing is defined by:

\begin{center}
\textbf{A manufacturing process where raw materials and energy are consumed in a continuous stream.}
\end{center}

A continuous manufacturing could be a chemical lab, where different chemicals are mixed in order to achieve a specific chemical. If something goes wrong, or the ratio is wrong, the process can't be undone and reversed into the starting chemicals.

This process cannot, in most situations, be paused and resumed at will, as the chemical reaction is active at all times.
\end{flushleft}
